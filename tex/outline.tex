\documentclass[iop]{emulateapj}

\usepackage{amsmath,amssymb}
\usepackage{color}
\usepackage{natbib}
\usepackage{graphicx}
\usepackage{hyperref}
\usepackage{ulem}
\usepackage[draft]{todonotes}

\newcommand\toplotrm[1]{\todo[color=green, inline, size=\small]{Plot: #1}}
\newcommand\towriterm[1]{\todo[color=yellow, inline, size=\small]{Write: #1}}
\newcommand\todorm[1]{\todo[color=cyan, inline, size=\small]{To do: #1}}

\newcommand\toplotemh[1]{\todo[color=pink, inline, size=\small]{Plot: #1}}
\newcommand\towriteemh[1]{\todo[color=orange, inline, size=\small]{Write: #1}}
\newcommand\todoemh[1]{\todo[color=red, inline, size=\small]{To do: #1}}

\newcommand\towrite[1]{\todo[color=gray, inline, size=\small]{Write: #1}}

\citestyle{aa}

\shorttitle{Meta-Calibration} \shortauthors{Huff and Mandelbaum}

\begin{document}
\title{ Meta-Calibration: Direct Self-Calibration of Shear Measurement Biases}
\author{Eric M. Huff\altaffilmark{1}}
\author{Rachel Mandelbaum\altaffilmark{2}}

\altaffiltext{1}{Center for Cosmology and Astroparticle Physics, 
Department of Physics, The Ohio State University, OH 43210, USA}
\altaffiltext{2}{McWilliams Center for Cosmology, Department of Physics, Carnegie Mellon University,
  5000 Forbes Avenue, Pittsburgh, PA 15213, USA}

\keywords{cosmology: observations --- gravitational lensing: weak ---
  methods: observational}

\begin{abstract}
\towriteemh{Pithy yet exciting abstract.  Mention public code.}
\end{abstract}

\maketitle

\towriteemh{Eric}
\todoemh{Eric}
\toplotemh{Eric}
\towriterm{ Rachel}
\todorm{Rachel}
\toplotrm{Rachel}
\towrite{Either of us. Rachel should seize if she feels the impulse, otherwise Eric will take it.}

\section{Introduction}
\towriteemh{ Weak lensing is important because...}
\towriteemh{ Lensing measurement is an unsolved problem because of the following outstanding measurement problems.}
\towriteemh{ We propose a solution that deals with measurement problems in a nicely general way.}


\section{Method}
\towriteemh{Motivation -- it is mechanically easier to introduce additional lensing or psf distortion into the images than it is to reliably measure those distortions at the necessary precision. This is why we turn to simulations for shear calibration in the first place.}

\subsection{Generating Counterfactual Images}
\subsubsection{Derivation}
\towriteemh{Derive the metacalibration procedure}
\towriteemh{Write the objective function for the counterfactual}
\towriteemh{Show that MetaCalibration maximizes the objective}
\towriteemh{Discuss noise amplification, and procedure for removing it.}

\subsubsection{Implementation}
\towriterm{implementing the above procedure requires precise implementation of X.  
  GalSim has been shown to do X... (i.e. reassure the reader that algorithmic precision is not an issue).}
\towriterm{what, specifically, we do with GalSim to make the counterfactual images}
\toplotrm{multi-panel galaxy images depicting counterfactual images}
\towriteemh{Using counterfactual images to take numerical derivatives of a black-box measurement pipeline}
\towriteemh{...with respect to shear}
\towriteemh{...with respect to psf}
\towriteemh{relate these derivatives to the calibration parameters in the larger shear measurement problem.}
\towriteemh{Note that these derivatives are very noisy, so results are sensitive to how we average.}

\subsection{Meta-Calibrating the Ensemble}
\subsubsection{the Estimator}
\towriteemh{Choosing a data vector (i.e., why histogram instead of averaging over objects, why choose the bins we do...)}
\towriteemh{Deriving the estimator (write the expression, including the derivatives)}
\towriteemh{relate the expressions in the estimator to the derivatives from the previous section}
\toplotemh{Show plots describing the data vector and estimator, for our three different shape measurement methods.}
\towriteemh{tunable parameters: binning scheme, prior}
\towriteemh{define the likelihood, motivate its use as a quality metric}
\towriteemh{motivate method for setting tunable parameters}

\section{Testing Framework}
\subsection{Great3}
\towriterm{brief-ish description of GREAT3 sims and sim framework. Basically, give enough info
 to show what’s going on and convince the reader that this is a non-trivially complicated dataset in 
terms of the PSFs and galaxy samples. Also note difference from real life in terms of expected mean shear, 
since that’ll be important later.}

\subsection{Estimation Methods}
\towriterm{We picked three trivially available estimation methods. Briefly describe each.}

\subsection{Simulation Branches}
\subsubsection{CGC without Aberrations}
\towriterm{Describe this branch}
\towriterm{Describe performance expectations (i.e., why is this the easiest) }
\toplotemh{Likelihood histogram, mean shear vs. cutoff and nbins}
\todoemh{Generate KSB results for this field}
\todoemh{Generate Moments results for this field}
\toplotemh{Regauss, KSB, and Moments overall performance on this branch}
\towriteemh{Briefly summarize calibration results for the three methods here.}
\subsubsection{RGC without aberrations}
\towriterm{Describe this branch}
\towriterm{Describe performance expectations (i.e., why is this probably very easy) }
\toplotemh{Likelihood histogram, mean shear vs. cutoff and nbins}
\todoemh{Generate KSB results for this field}
\todoemh{Generate Moments results for this field}
\toplotemh{Regauss, KSB, and Moments overall performance on this branch}
\towriteemh{Briefly summarize calibration results for the three methods here.}
\subsubsection{CGC Great3}
\towriterm{Describe this branch}
\towriterm{Describe performance expectations (i.e., why is this hard) }
\toplotemh{Likelihood histogram, mean shear vs. cutoff and nbins}
\todoemh{re-Generate KSB results for this field}
\todoemh{re-Generate Moments results for this field}
\toplotemh{Regauss, KSB, and Moments overall performance on this branch}
\towriteemh{Briefly summarize calibration results for the three methods here.}
\subsubsection{RGC Great3}
\towriterm{Describe this branch}
\towriterm{Describe performance expectations (i.e., why is this hard) }
\toplotemh{Likelihood histogram, mean shear vs. cutoff and nbins}
\toplotemh{Regauss, KSB, and Moments overall performance on this branch}
\towriteemh{Briefly summarize calibration results for the three methods here.}
\subsubsection{RGC or CGC with constant aberrations}
\todorm{Simulate this branch}
\todorm{Get calibration results for this branch}
\towriterm{Describe this branch}
\towriterm{Describe performance expectations (i.e., why is this one easy for MetaCal despite bad overall image quality) }
\toplotemh{Likelihood histogram, mean shear vs. cutoff and nbins}
\toplotemh{Regauss, KSB, and Moments overall performance on this branch}
\towriteemh{Briefly summarize calibration results for the three methods here.}


\section{Results}
\towriteemh{Brief outline of overall results.}
\toplotemh{Table of shear and psf calibration biases}
\toplotrm{Something along the lines of: Great3-style plot showing before/after performance in some projection of m/c/a for regauss/ksb/possibly moments. Something that can easily be compared with one or more figures in the Great3 results paper.}


\section{Applicability to Real Data}
\towriteemh{Comment on effects that this method may deal with, which we haven't tested (selection biases, other systematics detrending)}
\towriteemh{Comment on what measurements this can be straightforwardly applied to (mass mapping, g-g lensing) and what needs further thought (shear-shear) prior to deployment}
\towriteemh{Comment on effects that this method may or may not play well with -- blending, detector effects, nonlinearity, star-galaxy confusion}
\towriteemh{Comment on noise symmetrization -- we tried this, we don't understand why it didn't work as anticipated, but it doesn't seem to be important at this level of precision.}
\towriteemh{Put computational and implementation difficulty (for a realistic survey pipeline) in context. Is this substantially harder than making loads of independent sims? What are the prospects for further optimization, or other shortcuts?}

\section{Discussion and Conclusions}
\towrite{Put performance measures from previous sections in context. How do these results compare with the state of the art? Will this be good enough for DES, Kids, LSST?, SuMiRE?}



\appendix

\section{Code Release}
\towrite{Implementation details here. Not actually sure who should write this.}

\end{document}