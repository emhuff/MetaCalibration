\documentclass[iop]{emulateapj}

\usepackage{amsmath,amssymb}
\usepackage{color}
\usepackage{natbib}
\usepackage{graphicx}
\usepackage{hyperref}
\usepackage{ulem}
\usepackage[draft]{todonotes}

\newcommand\toplot[1]{\todo[color=green, inline, size=\small]{Plot: #1}}
\newcommand\towrite[1]{\todo[color=yellow, inline, size=\small]{Write: #1}}
\newcommand\todorm[1]{\todo[color=cyan, inline, size=\small]{To do: #1}}

\citestyle{aa}

\shorttitle{Meta-Calibration} \shortauthors{Huff and Mandelbaum}

\begin{document}
\title{How to Eliminate Multiplicative Shear Biases}
\author{Eric M. Huff\altaffilmark{1}}
\author{Rachel Mandelbaum\altaffilmark{2}}

\altaffiltext{1}{Center for Cosmology and Astroparticle Physics, 
Department of Physics, The Ohio State University, OH 43210, USA}
\altaffiltext{2}{McWilliams Center for Cosmology, Department of Physics, Carnegie Mellon University,
  5000 Forbes Avenue, Pittsburgh, PA 15213, USA}

\keywords{cosmology: observations --- gravitational lensing: weak ---
  methods: observational}

\begin{abstract}
\towrite{Pithy yet exciting abstract.  Mention public code.}
\end{abstract}

\maketitle

\todorm{discuss title.  I do not like the above, because (a) this is about a lot more than
  multiplicative bias, (b) there are lots of ways to ``eliminate'' biases that are less appealing.
  (Technically, you remove them if you make a fake dataset and test for calibration biases.  You
  just might not remove them very well.)  How about ``MetaCalibration: direct self-calibration of
  weak lensing shear biases''?}
\todorm{make writing assignments for each section.}

\section{Introduction}

\towrite{Weak lensing is awesome, but systematic errors in shear estimates are a long-standing
  problem in which much progress has been made in the past 2 decades.  Algorithms are not
  demonstrably ready for stage-IV data, and often must rely on external simulations for
  calibration.  This is unsatisfying because INSERT REASONS.  We propose a method for
  self-calibrating multiplicative biases, additive biases, and selection biases directly from the
  data through constrained re-simulation of the dataset.  This automatically uses the right galaxy
  population and PSF to describe the survey.  Isn't that cool?}

\section{Method}

\towrite{Brief conceptual description.  Need for a good model of what biases depend on.}

\subsection{Resimulation}

\towrite{thorough description of the resimulation aspect of this.  Need to talk about noise
  asymmetry, the expected impact, what one could do about this.}

\subsection{Shear inference}

\towrite{thorough description of the shear inference aspect of this.  Emphasize that we can't expect
reliable per-object derivatives when the objects are so noisy, so we have to work with ensemble
quantities to get effective average responses.}

\subsection{Outlier rejection}

\towrite{Consequence of using global prior, how to do rejection without accessing truth data.
  Comment on variances for outliers not being helpful.}

\subsection{Setting adjustable parameters of the method}

\towrite{Talk about likelihood cutoff, number of bins.  Explain how to set them without using the
  truth table, which is important for being able to use this in real life.  Do we have an
  understanding of the biases that result from using wrong $N_\text{bins}$?}

\section{Shear estimation methods}

\towrite{Since MetaCalibration can wrap essentially any per-object shear estimation method, we
  should valid it on more than one.  We will use re-Gaussianization and KSB.  For fun, we will also
  check how it does with uncorrected moments: can MetaCalibration due the entire PSF correction?
  It's not clear that the linear model should hold, but let's try anyway!  This is kind of an
  extreme case.}

\section{Test datasets}

\towrite{brief-ish description of GREAT3 sims and sim framework.  Basically, give enough info to
  show what's going on and convince the reader that this is a non-trivially complicated dataset in
  terms of the PSFs and galaxy samples.  Also note difference from real life in terms of expected
  mean shear, since that'll be important later.}

\section{Results}

\towrite{brief outline of results presentation goes here}
\todorm{decide if there are more calculations we want to do.  Example: should we make a larger
  simulated dataset to drive down our errors on $m$, $a$, $c$?}

\subsection{GREAT3 CGC}

\toplot{Per-object responsivities and so on.  Basically, show that this is noisy as heck.}
\toplot{Mean residual shear across the field as a function of number of bins, likelihood cutoff.}
\toplot{Ellipticity prior for the entire branch?}
\towrite{Discuss findings for number of bins, likelihood cutoff.}  
\toplot{Multiplicative, additive biases for number of bins, likelihood cutoff.  Highlight the point
  that was chosen in the previous plot.}
\towrite{Give $m_{1,2}$, $a_{1,2}$, $c_{1,2}$ for the chosen
  settings in the text, with errorbars.  Compare with $m_{1,2}$, $a_{1,2}$ for re-Gaussianization
  from GREAT3 leaderboard to assert that this has done a lot of the work for us.}
\towrite{Comment on outliers.  What fraction, how does that correlate with the PSF properties.  This
  will motivate next subsection.}

\subsection{GREAT3 CGC without optical aberrations}

\toplot{Mean residual shear across the field as a function of number of bins, likelihood cutoff.}
\towrite{Discuss findings for number of bins, likelihood cutoff.}  
\toplot{Multiplicative, additive biases for number of bins, likelihood cutoff.  Highlight the point
  that was chosen in the previous plot.}
\towrite{Give $m_{1,2}$, $a_{1,2}$, $c_{1,2}$ for the chosen
  settings in the text, with errorbars.  Comment on reduced outlier fraction, and why we think
  aberrations might have caused this.  (Model failure?)}
\todorm{Anything else to do on this?  Should we make a CGC that has some non-negligible set of
  aberrations that is present in each field?}

\subsection{GREAT3 RGC}

\towrite{Use a branch with realistic galaxy population as another check on the method.}
\toplot{The usual stuff.  Then discuss it in the text.}

\subsection{Other shear estimation methods}

\todorm{Put together KSB results for CGC, RGC?  Would be good to do both.}
\toplot{Some relevant KSB plots.}
\towrite{Comment on KSB results.}
\todorm{Put together moments results for CGC, RGC?  Would be good to do both.}
\toplot{Some relevant moments plots.}
\towrite{Comment on moments results.}

\subsection{Noise symmetry}

\towrite{Explain what happens when we symmetrize the noise.  It gets a lot noisier, which is not
  quite expected.}
\todorm{Consider whether more investigation is needed on this.  Can we explain (e.g., using Chris's
  calculations) why it seems to be too small to matter?}

\section{Applicability to existing data}

\towrite{This is constant shear.  Could this be used for g-g lensing? Discuss limitations.}
\towrite{What is needed for variable shear?}
\towrite{Further extensions: selection bias?}
\towrite{Inclusion of weights?  Currently individual objects are not weighted.  Discuss.}
\towrite{Comment on data processing and image processing requirements, compared to other ways of
  calibrating shears (e.g., making loads of external sims).  Can we do some stupid optimization,
  e.g., use a subset of the survey (chosen at random)?}

\section{Discussion and conclusions}

\towrite{We win.}

\section*{Acknowledgments}

\towrite{Funding acknowledgments, discussions.}

\bibliographystyle{apj}
\bibliography{bibliography}

\appendix

\section{Publicly available implementation}

\towrite{Implementation details here.}
\todorm{Clean up enough that we can make it public!}

\end{document}
