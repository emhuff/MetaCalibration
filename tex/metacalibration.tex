\documentclass[iop]{emulateapj}

\usepackage{amsmath,amssymb}
\usepackage{color}
\usepackage{natbib}
\usepackage{graphicx}
\usepackage{hyperref}
\usepackage{ulem}


\citestyle{aa}

\shorttitle{Meta-Calibration} \shortauthors{Huff and Mandelbaum}

\begin{document}
\title{How to Eliminate Multiplicative Shear Biases}
\author{Eric M. Huff \altaffilmark{1}}
\author{Rachel Mandelbaum \altaffilmark{2}}

\altaffiltext{1}{Center for Cosmology and Astroparticle Physics, 
Department of Physics, The Ohio State University, OH 43210, USA}
\altaffiltext{2}{McWilliams Center for Cosmology, Department of Physics and Astronomy, Carnegie Mellon University, PA 11111, USA}

\keywords{cosmology: observations --- gravitational lensing: weak ---
  methods: observational}

\begin{abstract}
  One of the major limiting sources of systematic error in forthcoming weak lensing measurements is
 systematic uncertainty in the quantitative relationship between the distortions due to gravitational lensing and 
 the measurable properties of galaxy images. We present a statistically principled, general solution to this 
problem. Our technique infers calibration parameters for an arbitrary shape measurement technique by modifying 
the real images to simulate the effects a known shear. We test our results on simulated images from the 
Great3 shear calibration challenge, and show that the method eliminates calibration biases for a variety 
of shape measurement techniques  at the level of precision measurable with the available Great3 simulations.
\end{abstract}


\section{Introduction}
Accurate measurement of weak gravitational lensing offers the most direct probe of the dark sector of the universe [REF: Weinberg et al 2013]]. Several billion dollars' worth of ambitious ground- and space-based imaging surveys currently under way plan to make this measurement, but reliable measurement of this effect is not yet a solved problem. There are several reasons for this.
Shear elongates the shapes of galaxies along lines of constant projected gravitational potential. A realistic shear arising from large-scale structure changes the axis ratio of an idealized galaxy by less than $1\%$. For Euclid, LSST, or WFIRST, the statistical errors will be of order $10^{-3}$ of this small signal. Several observational effects, like the smearing of ground-based images due to the finite telescope optics and the atmosphere, produce shape distortions to the images of galaxies that are coherent at the same angular scale as and orders of magnitude larger than the statistical error budget of these experiments.  Even with an accurate model for these effects, existing measurement techniques can fail due to ignorance of the statistical distribution of unlensed galaxy shapes [REF]. For the faint, poorly-resolved, and low signal-to-noise galaxy images that comprise the bulk of the sample of any lensing survey, most shear estimation methods experience noise biases and selection biases which can swamp the statistical errors [REF].

The weakness of the signal in the cosmological regime permits accurate linear parameterization of the relationship between the true $(\gamma)$ and estimated ($\hat{\gamma}$) shear, of the form
\begin{align}
\hat{\gamma} = (1+m)\gamma + c
\end{align}

In the presence of these effects, a weak lensing measurement must suppress systematic errors in $m$ and $c$ below $10^{-3}$ and $10^{-3.5}$, respectively [REF: Huterer et al 2006]. A great many implementations of shear measurement techniques appear in the literature [REF$\times$many], and the community has undertaken to test the performance of many of these methods with a series of blind simulation challenges [REF: STEP, GREAT]. The realism of these challenges was restricted in order to focus on treating specific systematic error sources. The success in meeting the above shear calibration requirements demonstrated by several teams in the Great3 challenge 
 [REF: Great3 results paper] represents important progress that is nevertheless qualified by the deliberately limited range of realism of the Great3 image simulations. 

Our strategy is to deploy a meta-pipeline, which wraps around an arbitrary shear detection and measurement pipeline such as those currently implemented for real measurements and the shear challenge simulations. We show how to perturbatively modify the data to simulate additional shear, psf distortions, and other similar effects, and then make measurements on these perturbed images using a variety of existing measurement methods. We show that this process permits accurate estimation and de-trending of the linear shear response and additive biases, including psf effects. While shear- or psf-dependent selection biases are not within the scope of the available simulations, we expect that this technique can accurately account for both.

\section{Theory}
The shear sensitivity is the quantitative relationship between the measured properties of galaxies in the sample under consideration and the shear component of the weak lensing distortion field. For some analysis methods, only the ensemble-averaged calibration is required [REF: lots and lots, starting with KSB]. Some approaches [REF: LensFit, by Miller et al.] derive a shear sensitivity for each galaxy. Others require reliable prior information about the ensemble distribution of galaxy properties [REF: Bernstein \& Armstrong, Sheldon]. Selection biases, where shear alters the composition of the analysis sample, can be as large as several per cent of the shear signal [REF:Hirata\&Seljak 2004, maybe others].

Fortunately, for the weak shears under consideration in most cosmological survey applications, the relationship between the shear and the galaxy shapes (or related observables) is linear, so each of these effects ends up modifying the first derivative of the galaxy properties with respect to the shear. What follows is a method for estimating this derivative directly from the images. Throughout we will assume that the observed image $I({\mathbf{x}})$ is the unsmeared galaxy image $G(\mathbf{x})$ convolved with some seeing kernel $P(\mathbf{x})$.

In an ideal world, we would vary the gravitational shear experienced by the image before is smeared by $P$, constructing the counterfactual image $I'(\mathbf{x}| {\boldsymbol \gamma})$:
\begin{equation}
  I'({\mathbf{x}}) = P \otimes\left( \hat{\mathbf{s}}G\right)
\end{equation}

where $\hat{\mathbf{s}}$ is the shear operator that produces the shear ${\boldsymbol \gamma}$, as in e.g. [REF:BJ02]. The shear sensitivity would then be a straightforward numerical derivative of $I'$ with respect to ${\boldsymbol \gamma}$. We can even write down a procedure for producing $I'$ from $I$ if we understand $P$:
\begin{equation}
  I'({\mathbf{x}}) = P \otimes \left[ \hat{\mathbf{s}}\left( P^{-1} \otimes I \right)\right].
\end{equation}

The noise in $I$ has non-zero power on scales where $P$ is small or vanishing. Deconvolution amplifies noise, and because of the shear this is not cancelled by reconvolution with $P$. 

The noise amplification can be mitigated by reconvolving after the shear operation with a new psf $\Gamma$, (instead of $P$) and constructing $\Gamma$ so that it suppresses the noise amplification that would normally be produced by the deconvolution operation. All that is required for this is that (in Fourier space, with the tilde indicating the Fourier transformed quantity) $\|\tilde{\Gamma}(\mathbf{|k|}) \| \geq \|\hat{\mathbf{s}} \tilde{P}(\mathbf{k})\|$ for all $\mathbf{k}$, which can be met without introducing additional PSF anisotropy by choosing $\Gamma(\mathbf{x}) = P\left((1+2|\gamma|)\mathbf{x}\right)$.

Formally, then, we seek a procedure, denoted here by the operator $\hat{A}$, that minimizes the difference between the image we have and a sheared version of the same image with a different psf. This quantity is:
\begin{equation}
{\cal E}_1 = \int \| \Gamma \otimes \left[\hat{\mathbf{s}} I(\mathbf{x})\right] - \hat{\mathbf{A}} I \|
\end{equation}
and the operator $\hat{\mathbf{A}} $ that extremizes ${\cal E}_1$ is:
\begin{equation}
\hat{\mathbf{A}}  = \Gamma \otimes \left[ \hat{\mathbf{s}} \left(P^{-1} I \right)\right].
\end{equation}

This procedure clearly requires a good model for $P$, but so do all shear measurements. PSF model mis-specification errors enter at the same order in measurements on the resulting image that they would in an unmodified image. [NOTE: Do we learn anything from the noise properties of shape measurements on $I'$ about whether we've inadvertently amplified the noise by de-convolving with an artificially small PSF?].

Once the counterfactual image $I'(\mathbf{x}|\gamma)$ with $\|{\boldsymbol \gamma}\| << 1$ has been created, the galaxy detection and  shear measurement pipeline should be run. This provides a measure of the shear sensitivity for an image with the PSF $\Gamma$, not an image with the psf $P$. This requires that the full measurement -- not just the sensitivity analysis -- be run on a third image $I'(\mathbf{x}|\gamma=0)$, so that the numerical derivative $\frac{\partial I'}{\partial \gamma}$ is well-defined. 

This procedure introduces correlated, anisotropic noise, which can produce a systematic multiplicative shear bias. If the noise properties of the initial image are known, the noise anistropy can be removed with additional anisotropic noise. As we describe below, we have not found noise isotropization to be a necessary step.

MetaCalibration can be used to mitigate other systematics as well. Even those measurement methods with the highest scores in the Great3 lensing challenge were unable to completely remove the effects of psf ellipticity on the inferred shear. We show below that reconstructing images with added PSF ellipticity, rather than added shear, allow us to de-trend the effects of psf anisotropy. A similar approach could be used any effect -- signal or systematic error --  which can be simulated by perturbing the images as above.

\section{Implementation}
We have created a simple pipeline that takes as inputs postage stamps of the galaxy and psf model, and returns a set of modified images, as described below. A shape measurement code -- the details of which MetaCalibration is agnostic about -- returns shear or shape estimates for each of the modified images. The resulting set of shapes is used to derive calibration and psf biases for each galaxy. These parameters, along with a shape prior inferred from the full ensemble of shapes, are used to derive a mean shear per field. Virtually any measurement method can be embedded in this loop, and as long as it is sensitive to the shear and not catastrophically biased, the linear shear and psf calibration biases will be removed. MetaCalibration can also calibrate away shear selection biases, so long as detection is performed inside the procedure.

\subsection{Image Modification}
We use GalSim[REF:GalSim] \footnote{\url{https://github.com/GalSim-developers/GalSim}} to manipulate the images and to generate simulations for validation. For each galaxy, we create nine modified images: two for each shear component, two for each psf ellipticity component, and one for the final measurement. We run the provided shape measurement pipeline on each of these images, and the results are used to construct a set of finite difference estimates of calibration and psf biases.

This sort of image manipulation is very similar the simulation design goals of the GalSim  project, so we rely on the rigorous testing of the image convolution, interpolation, and resampling algorithms the development team performed to enable the Great3 shear testing simulations.

For each galaxy and psf postage stamp, we first create an Interpolated Image object. This object is deconvolved by the psf model (including the pixel response). For the shear finite differences, we apply a small shear $\Delta\gamma$ (typically 1\%) to the resulting deconvolved image. The original psf is dilated by twice the shear distortion, and then re-convolved with the sheared deconvolved image. This reconvolved, sheared image is then passed to the shape measurement routine, along with the newly dilated psf. For the psf sensitivity, we follow a similar procedure, but shear the dilated psf image, rather than the deconvolved galaxy image. Finally, we create a reconvolved image with no added shear, on which we'll perform the final shape measurement. {\bf EMH: We should find more compact nomenclature for these images}.

Shape measurements on these images are used to derive shear calibration and psf biases {introduced by the chosen shape measurement method}. The shapes measured from the  sheared reconvolved images, $\vec{e}_{+}$ and $\vec{e}_{-}$, admit a straightforward finite-difference estimate of the multiplicative shear calibration
\begin{align}
R &= \frac{\partial \vec{e}}{\partial \vec{\gamma}}  \\
 &=\frac{\vec{e}_{+} - \vec{e}_{-}}{2\Delta\gamma}
\end{align}
Additive biases introduced by the shape measurement are related to the sum of these two quantities:
\begin{align}
\vec{c} &= \frac{\vec{e}_{+} + \vec{e}_{-}}{2 \Delta\gamma} - \vec{e}
\end{align}
and if the shape measurement algorithm does not perfectly remove psf ellipticity, then the shapes measured from shearing the psf ($\vec{e}_{+,\rm psf}$ and $\vec{e}_{+,\rm psf}$) allows calculation of at least the linear-order psf ellipticity biases:
\begin{align}
R_{\rm psf} = \frac{\vec{e}_{+\rm psf} - \vec{e}_{-,\rm psf}}{2\Delta\gamma}.
\end{align}
The result of this is a catalog of shear responsibities, psf responsivities, and additive biases for every galaxy.


\subsection{Shear Inference}
We test the MetaCalibration procedure on two different shear estimation methods -- {\sc regauss} and {\sc KSB} -- each of which has known calibration biases [REF; also see figure XXX]. For each of these methods, we use the entire ensemble of validation simulations to build a prior the {\it unlensed} shape distribution, $p_0(\vec{e})$. There is no guarantee that the average shear over the ensemble is actually sufficiently small, however, so we symmetrize this prior distribution by averaging the raw prior with its reflection about $\vec{e}=0$. The newly symmetrized prior is $p_{0,\rm sym}$. The model for each field is
\begin{align}
\vec{e}_{\rm meas} = \vec{e}_{0} + R_{\rm psf} \vec{e}_{\rm psf} + R\vec{\gamma} + \vec{c}
\label{eqn:edist_model}
\end{align}
where $e_{\rm meas}$ is the vector of measured ellipticities, and the constants $R_{\rm psf}$, $R$, and $c$ have been determined separately for each galaxy, as described above, during the image modification step. This prior is then used to construct a linear estimator for the shapes, as follows.

If the measured shape distribution $n(\vec{e}_{\rm meas})$ is linear in the shear, then we can write
\begin{align}
\frac{n(\vec{e}_{\rm meas})}{N_{tot}} = p_{0,\rm sym}(\vec{e}) + \vec{\gamma}\cdot \partial_{\gamma} p_{0,\rm sym}(\vec{e})
\end{align}
It will be convenient to discretize this distribution into a histogram. If the probability of a galaxy ending up in the $i^{\rm th}$ shape histogram bin is $q_i$, then the likelihood function for an observed histogram is exactly the multinomial likelihood
\begin{align}
p( \left\{N_i\right\} |\left\{ q_i\right\} ) = \frac{N_{\rm tot}!}{\prod\limits_i (N_i!)}\prod\limits_j q_j^{N_j}
\label{eqn:multnomial}
\end{align}
where $N_{\rm tot} = \sum\limits_i N_i$ is the total number of samples in the histogram.
The covariance matrix for the bin amplitudes this histogram is
\begin{align}
{\rm cov}(N_i, N_j) = C_{ij} = \begin{cases}
  q_i(1-q_i)\sum\limits_i N_i, & i=j \\
  -q_iq_j \sum\limits_i N_i, & i \neq j.
\end{cases}
\end{align}
To make the notation for what follows less cumbersome, let the normalized histogram be $h_i = N_i / \sum\limits_i N_i$, and its first derivative with respect to the shear be $\vec{\Delta}_h=\partial_{\gamma}\vec{h}_{\rm fid}$.

Given a measured shape histogram with some unknown shear and a fiducial, unlensed shape histogram, the (component-wise) minimum-variance estimator for $\gamma$ is
\begin{align}
\hat{\gamma} = \frac{\vec{\Delta}_h^T C^{-1}\left( \vec{h}_{\rm meas} - \vec{h}_{\rm fid}\right)} {\vec{\Delta}_h^TC^{-1}\vec{\Delta}_h},
\label{eqn:hist_est}
\end{align}
and it has variance
\begin{align}
\sigma^2_{\hat{\gamma}} = \frac{1}{\vec{\Delta}_h^TC^{-1}\vec{\Delta}_h}
\label{eqn:hist_est_var}
\end{align}

This method for shear inference has as its tunable parameter only the histogram binning scheme, about which more below. Once we've chosen a suitable scheme, we then bin the prior into equal-number bins and calculate its shear derivative using equation \ref{eqn:edist_model}\footnote{We add a small shear $\vec{\gamma}$, then rebin.}. We calculate a shape histogram with these bins for each separate field, and evaluate equations~\ref{eqn:hist_est} and \ref{eqn:hist_est_var}.

If we have a poor model for the unlensed histogram, $h_{\rm fid}$, then the results will be biased. We can evaulate the distance from each field to the unlensed prior using equation~\ref{eqn:multnomial}, taking the probabilities $q_i$ from the unlensed prior and the histogram amplitudes from the current field, {\it after correcting for the estimated shear}. If the shear response measured for the unlensed prior is correct, then the performance of the estimator will depend only on the similarity of the prior t the measurement field. The likelihood can then be used as an objective criterion for the quality of the inference. 

\section{Results}



MetaCalibration performs well with all of the shear measurement methods tested. The regauss, KSB, linear, and moments measurements are all successfully calibrated, with multiplicative and additive biases removed, and psf ellipticity correlations nulled. The principal difference between methods appears to be the scatter around trend -- several times higher for KSB and linear mmoments than for re-gaussianization, for example.

The Python scripts that implement MetaCalibration as described here require $\sim1$s per galaxy image, but we have not thoroughly explored the possibilities for optimization. For Stage-IV Dark Energy lensing survey catalogs, with $\sim10^9$ distinct galaxy images, this is expensive but not prohibitive relative to the computing resources currently allocated to these projects.

\subsection{}




\bibliographystyle{apj}
\bibliography{bibliography}


\end{document}